\documentclass[12pt, letterpaper]{article}
\usepackage[margin=1in]{geometry}
\usepackage{amsmath,amsthm,array, amssymb,amsfonts, cancel, enumitem, fancyhdr, color, comment, graphicx, environ}

\setlist[description]{leftmargin=\parindent,labelindent=\parindent}

\author{Val Anthony Balagon}
\date{February 2019}
\title{Chapter 6: Eigenvalues and Eigenvectors}

%User commands
\newcommand{\R}[1]{$\mathbb{R}^{#1}$}
\newcommand{\Vector}[1]{$\textbf{#1}$}
\newcommand{\V}[1]{\textbf{\textit{#1}}}

\newcommand{\DefinitionSpace}{\vspace{15px}}


\newtheorem*{remark}{Remark}

\theoremstyle{definition}
\newtheorem{definition}{Definition}[section]

\newtheorem{example}{Example}
\newtheorem{theorem}{Theorem}

\newcommand*{\vertbar}{\rule[-1ex]{0.5pt}{2.5ex}}



\newenvironment{problem}[2][Problem]{\begin{trivlist}
		\item[\hskip \labelsep {\bfseries #1}\hskip \labelsep {\bfseries #2.}]}{\end{trivlist}}



\begin{document}
	\maketitle
	\begin{abstract}
		This chapter focuses on eigenvalues and eigenvectors.
	\end{abstract}

The system $A\V{x} = \V{b}$ is in equilibrium and steady state. Change as in time enters the picture - continuous time in a differential equation $\frac{d\V{u}}{dt} = A\V{u}$ or time steps in a difference equation $\V{u}_{k+1} = A\V{u}_k$. Using linear algebra, eigenvalues and eigenvectors allow these types of systems to be solved beautifully.

Vectors $\V{x}$ when multiplied by $A$ usually change direction. But there are certain exceptional vectors that maintain the same direction as $A\V{x}$ and these are called ``eigenvectors."




\end{document}