\documentclass[letterpaper]{article}
\usepackage[margin=1in]{geometry}
\usepackage{amsmath,amsthm,amssymb,amsfonts, enumitem, fancyhdr, color, comment, graphicx, environ}


\author{Val Anthony Balagon}
\date{January 2019}
\title{Chapter 3: Vector Spaces and Subspaces}

%User commands
\newcommand{\R}[1]{$\mathbb{R}^{#1}$}
\newcommand{\Vector}[1]{$\textbf{#1}$}
\newcommand{\V}[1]{\textbf{\textit{#1}}}
\newtheorem*{remark}{Remark}

\newenvironment{problem}[2][Problem]{\begin{trivlist}
		\item[\hskip \labelsep {\bfseries #1}\hskip \labelsep {\bfseries #2.}]}{\end{trivlist}}
\newenvironment{sol}
{\emph{Solution:}
}
{
	\qed
}
\specialcomment{com}{ \color{blue} \textbf{Comment:} }{\color{black}} %for instructor comments while grading
\NewEnviron{probscore}{\marginpar{ \color{blue} \tiny Problem Score: \BODY \color{black} }}


\begin{document}
	\maketitle
	\begin{abstract}
		This chapter focuses on vector spaces and subspaces. Topics include vector spaces and subspaces such as the column space $C(A)$, nullspace $N(A)$, and the rest of the \emph{four subspaces}.
	\end{abstract}


\section{The Nullspace of A: Solving $A\textbf{x} = \textbf{0}$ and $R\textbf{x} = \textbf{0}$}
	The nullspace is a subspace of matrix $A$ (square or rectangular) that contains all of the solutions to $A\textbf{x} = \textbf{0}$. A readily available solution to the system is the zero vector \Vector{Z}. Invertible matrices only have the zero vector as a solution while non-invertible matrices have nonzero solutions. \textbf{Each solution \Vector{x} belongs to the nullspace of $A$ which is in \R{n}.}
		
	\begin{problem}{1} 
				Describe the nullspace of the singular matrix $A = \begin{bmatrix}
																		1 & 2 \\
																		3 & 6
																		\end{bmatrix}$.
	\end{problem}
	
	\begin{sol}
		We produce the following from elimination: 
		\begin{gather*}
			A \V{x} = 0  \\
			\begin{bmatrix}
			1 & 2 \\
			0 & 0 
			\end{bmatrix}\begin{bmatrix}
								x_1 \\
								x_2
						\end{bmatrix} =	\begin{bmatrix}
												0 \\
												0 
											\end{bmatrix}
			\intertext{\textbf{Elimination produces a single line}. This line is in the nullspace of $A$ and \textbf{contains all solutions ($x_1, x_2$)}. $x_2$ is a free variable. We choose $x_2 = 1$ as our special solution because all points on the line are multiples to it. From there, we get $x_1=-2$.}
				\V{s} = \begin{bmatrix}
						-2 \\
						1
						\end{bmatrix} \\
			\intertext{Hence,}
							A\V{s} = \begin{bmatrix}
												1 & 2 \\
												3 & 6
											\end{bmatrix} \begin{bmatrix}
															-2 \\
															1
															\end{bmatrix} = \begin{bmatrix}
																					0 \\
																					0
																					\end{bmatrix} \\
		\end{gather*}
	\end{sol}

	The nullspace of $A$ consists of all combinations of the special solutions to $A\V{x}=\textbf{0}$
	
	\begin{problem}{2} 
		$x+2y+3z=0$ comes from the $1 \times 3$ matrix $\begin{bmatrix}
															1 & 2 & 3
														\end{bmatrix}$. Then $A\V{x}=\textbf{0}$ produces a plane. All vectors on the plane are perpendicular to $(1,2,3)$. \textbf{The plane is the nullspace of A}. There are 2 free variables $y$ and $z$: Set to $0$ and $1$.
	\end{problem}

	\begin{sol}
			\begin{gather*}
				A \V{x} = \begin{bmatrix}
				1 & 2 & 3
				\end{bmatrix} \begin{bmatrix}
								x \\
								y \\
								z
								\end{bmatrix} = 0
				\intertext{We have two free variables $y$ and $z$, hence we have two special solutions. For $\textbf{\textit{v}}_1$, we choose $y=1$ and $z=0$.}
					x + 2(1) + 3(0) = 0 \qquad \rightarrow \qquad x = -2 \\
					\V{s}_1 = \begin{bmatrix}
								-2 \\
								1 \\
								0
							\end{bmatrix}
				\intertext{For $\V{s}_2$, we choose $y=0$ and $z=1$. }
					x + 2(0) + 3(1) = 0 \qquad \rightarrow \qquad x = -3 \\
					\V{s}_2 = \begin{bmatrix}
								-3 \\
								0 \\
								1
								\end{bmatrix}
				\intertext{Checking for $\V{s}_1$ and $\V{s}_2$,}
					A \V{s}_1 = \begin{bmatrix}
									1 & 2 & 3
									\end{bmatrix} \begin{bmatrix}
														-2 \\
														1\\
														0
														\end{bmatrix} = -2 + 2 + 0 = 0 \\
				A \V{s}_2 = \begin{bmatrix}
								1 & 2 & 3
								\end{bmatrix} \begin{bmatrix}
												-3 \\
												0 \\
												1
												\end{bmatrix} =	-3 + 0 + 3 = 0
			\end{gather*}
			The vectors $\V{s}_1, \V{s}_2$ lie on the plane $x + 2y + 3z=0$. All vectors on the plane are combinations of $\V{s}_1$ and $\V{s}_2$.
	\end{sol}

	\begin{remark}
		There are two key steps in finding the nullspace of a matrix. 
		\begin{enumerate}
			\item Reducing $A$ to \textbf{row echelon form $R$}
			\item Finding the special solutions to $A\V{x}=\textbf{0}$
		\end{enumerate}
	\end{remark}

\subsection{Pivot Variables and Free Columns}
	Free components correspond to columns with no pivots. The special choice (1 or 0) is only for the free variables in the special solutions.
	
		\begin{problem}{2} 
			Find the nullspaces of the following matrices.
			
			\begin{gather*}
					A =  \begin{bmatrix}
							1 & 2 \\
							3 & 8
						  \end{bmatrix} \qquad  B =  \begin{bmatrix}
													  A \\
													  2A
													  \end{bmatrix} =  \begin{bmatrix}
																			  1 & 2 \\
																			  3 & 8 \\
																			  2 & 4 \\
																			  6 & 16
																			  \end{bmatrix} \qquad  C =  \begin{bmatrix}
																											  A & 2A
																											  \end{bmatrix} =  \begin{bmatrix}
																																  1 & 2 & 2 & 4 \\
																																  3 & 8 & 6 & 16
																																  \end{bmatrix}
			\end{gather*}

	\end{problem}
	
	\begin{sol}
		\begin{gather*}
			\intertext{Elimination yields} 
				\begin{bmatrix}
					1 & 2 \\
					0 & 2
					\end{bmatrix} \begin{bmatrix}
									x_1 \\
									x_2
									\end{bmatrix} = \begin{bmatrix}
														0 \\
														0
														\end{bmatrix}
			\intertext{$A$ is an invertible matrix, hence \V{x} = $\begin{bmatrix}
																			0 \\
																			0
																			\end{bmatrix}$}
		\end{gather*}
	\end{sol}
	
	
\end{document}